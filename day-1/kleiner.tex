\addsection{Consciousness in Causal Cognition}

\sectionauthor[]{Johannes Kleiner}
\begin{affils}
  \sectionaffil[]{LMU Munich Center for Mathematical Philosophy (MCMP)}
  \sectionaffil[]{LMU Graduate School of Systemic Neurosciences (GSN)}
  \sectionaffil[]{Institute for Psychology, The University of Bamberg}
  \sectionaffil[]{Association for Mathematical Consciousness Science (AMCS)}
\end{affils}

Kleiner presents the claim that ``consciousness is relevant to, or part of, human
cognition''.
On this view, a proper understanding of cognition requires understanding of
consciousness, and proper modelling of cognition requires modelling of consciousness.
Historically, the notion of consciousness is strongly linked with the idea of `what it
is like' to be something (Nagel).
Wittgenstein: ostensive definition (pointing).

The `old-school' view of consciousness and its relation to cognition is that there are
incoming signals; some kind of `forward' cognitive processing; and conscious experience
is somehow produced `along the top'.
The `new-school' view is that cognitive processing is `bottom-up' as well as
`top-down'.
Active inference, Bayesian brain, predictive processing, free energy principle, etc.
See \textcites{Allen2018}{Tull2023}.
How do theories of consciousness impact this picture?
\begin{itemize}
  \item Global neuronal workspace theory: selects among multimodal parallel free energy
        subprocessors whose result is to distribute to all other processors.
  \item Higher order theories: monitoring, distinguishing, selecting among parallel free energy
        subprocessors (again) to make available to downstream processing.
  \item PP-consciousness (predictive-processing) proposals: consciousness is the generative
        model containing a model of itself and how itself interacts with the world.
        Self-evidencing, internal states and actions on the environment maximize the evidence
        for their own existence.
  \item Integrated information theory: consciousness is the causal structure that does the
        cognitive processing.
\end{itemize}

\paragraph{Questions}

Top-down versus bottom-up: we're only representing one part of the process in string
diagrams.
We think that systems show that systems can't be conscious.
But we're moving towards computation that doesn't have an instruction set that
determines the form of its computations.
