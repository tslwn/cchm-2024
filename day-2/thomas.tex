\addsection{Educational Neuroscience: Teaching the Brain to Understand Causality}[Educational Neuroscience]

\sectionauthor[]{Michael Thomas}
\begin{affils}
  \sectionaffil[]{Birkbeck, University of London}
\end{affils}

Educational neuroscience; the brain's `implementation constraints'; and an example of
translation(al) research: scientific concepts and inhibitory control.
What is educational neuroscience?
It's part of the new science of learning, which is an interdisciplinary field.
Psychology and education have a long history of interaction.
But the twin developments in neuroscience (\emph{in vivo} imaging) and machine learning
(computational POV) were new.
How do neuroscience and education interact?
Mostly within a cognitive neuroscience framework: improving psychological theories of
learning, which influence education in turn.
Possibility of a direct route: e.g., brain health, the brain as a metabolic device with
nutritional needs, stress response, etc. It's important to be modest as a
neuroscientist: classrooms are a complicated context.
We can't contribute to that much of it.
How much time is dedicated to engaged learning?
Constraints on learning in education -- Bronfenbenner's `Ecological Systems Theory' and
Michie et al.
`Behaviour Change Wheel'.
Hierarchy of constraints, e.g., individual attributes up to national policies.
50\% of the variation in learning outcomes is explained by genetics, school is about 10\% and
home/working environment is around twice that, i.e., there's only so much educational
neuroscience can do.

Why is it important to know how the brain works?
There are some unintuitive things about how the mind works: memory, framing,
plasticity.
There are other ways to run cognitive systems and control bodies (machines).
Our cognitive system works that way it does due to the brain (biology).
Biology works the way it does because of evolution.
Evolution only has locally optimal solutions.
Thinking with neurons might not be the best way: metabolic costs, etc. The brain isn't
like a digital computer (per 1980s cognitivism).
The brain has `content-specialised' devices.
A computer has a general code (abstraction) and domain-general processes.
The brain's knowledge is built into its structure -- you can't move information back
and forth between domain-general mechanisms.
Modulation and reconfiguration of content-specific circuits.
Hierarchies, hubs, maps, and networks.

How does the brain work?
The sensory system is a hierarchy of sensory modalities.
Coarse features at the bottom, higher-order invariances above.
Bidirectional connectivity.
E.g., sensory neurons do something like a deep convolutional neural network.
Then we have an association cortex to do mappings between them.
We're social, so a lot of what the brain is built for is social: expressions, motion,
space, intentions, narrative, etc. And then we have a motor hierarchy.
That similarly goes from immediate actions to sequences of actions, contingent actions,
and plans.
Might want to use perceptual information later.
A bunch of what the brain does is control: task schema selection and maintenance.
Valence map of emotional values to different associations.
Motor smoothing: 80\% of the brain's neurons!
Computationally, moving a chess piece is more complicated than deciding where to move
it in the first place.

Learning: one thing in the classroom, many things in the brain.
Multiple mechanisms, networks, roles of different factors.
Concepts also get messy from the POV of the brain.
Educational domains are combinations of sensorimotor operations, concepts, and
procedures.
E.g., mathematics combines symbols, words, bodily actions, amounts, facts,
calculations.
The tricky thing in neuroscience is translation, i.e., how do you translate insights
into brain implementation constraints into useful heuristics for teachers?
The brain's design priorities are sensorimotor, emotion, social, and cognitive -- in
that order.
To optimize learning, align the first three priorities with the fourth.
`Cognitive load theory' based on 1980s cognitive psychology.
Static memory, got to get knowledge through the bottleneck of working memory into
long-term memory.
Neuroscience you want to add other factors like executive functions, changes in
knowledge structures, agency, emotions, social context, adolescence, etc., which all
differ between individuals.

`Why don't kids like school?'
The brain continuously computes whether a task is worth the mental effort given the
expectations of success and reward.
Attention is metabolically expensive.
Learning scientific concepts.
Foundation of intuitive concepts since infancy.
New conceptual repertoire, abstractions, vocabulary, notation.
Transitions and transformations: temporal dimension.
Cognitive control/executive function; importance of language and group-based learning;
dissociations between prediction and explanation.
Sometimes facts are inconsistent or unintuitive.
Learning about maths and science sometimes involves inhibiting prior beliefs or direct
perceptual information.

Experts get better at inhibiting pre-potent responses rather than replacing prior
concepts with new ones.
Similar bits of the brain are activated in misconception-type problems.
The wider role of inhibitory control in reasoning.
Syllogistic reasoning with erroneous premises or counterfactual conclusions involves
cognitive control.
E.g., \emph{modus ponens}.
Use inhibition to focus on language properties of syllogisms rather than semantics
\parencites{Houde2015}.
The UnLocke project: large-scale neuroscience-based intervention study.\footnote{See
  \href{https://unlocke.org/}{\texttt{https://unlocke.org/}}} Key idea is to train
children to use existing inhibitory skills better in the context of math and science.
Content-specific circuits: not general skills but in the context of maths and science,
so need to do in that context.
Relation to dual-process theory.
Detail effect as it progresses across middle childhood.
Stop and Think paper.
Broader evidence-based approach to education.

Educational neuroscience uses understanding of implementation constraints to facilitate
classroom practices and improve learning outcomes.
Classroom learning involves many brain mechanisms, educational knowledge is hybrid.
Causal reasoning relates to sensorimotor biases, early developing of intuitive
knowledge, the importance of language to control attention and context, cognitive
control, etc. Maybe AI can do better than humans?
But probably also less well-adapted to human contexts.

\paragraph{Questions}

\begin{itemize}
  \item `Suspending judgment is the hallmark of morality'.
        Is it also ethical education?
  \item What about language?
        Complexities of language, expect some kind of neural correlates.
        Can't really find language in the brain as a structure, as opposed to other animals.
        Evolutionary POV.
  \item What do we want the educational system to do?
        Are you worried about the gaps between people rather than the mean?
        Seems easier to shift the whole distribution.
        Don't seem to know why Finland is good.
        Do those techniques for example generalize well to other populations.
  \item Domain-general versus -specific.
        Psychologists that talk about innate concepts need to explain how it's encoded in DNA
        etc. Something's innate if it's more robust to variation in environmental conditions.
        Pre-natal processes.
        Cortex is more generally plastic and has a lot of highly-conserved processes of
        activity-driven self-organization.
  \item Meaning-making and causal thinking emerges from social interaction.
        Sensorimotor bias to learn better if couched in that.
        Layer over abstract things.
        Basis for more abstract processes.
\end{itemize}
