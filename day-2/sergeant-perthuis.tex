\addsection{Challenges and Prospects for the Projective Consciousness Model}

\sectionauthor[]{Grégoire Sergeant-Perthuis}
\begin{affils}
  \sectionaffil[]{LCQB, Sorbonne Université}
\end{affils}

Phenomenological aspects of consciousness, computational phenomenology.
Consciousness involves a subjective perspective, characterized by viewpoint-structured
organization, a sense of unity (holistic), embodiment, and an internal representation
of the world in perspective from a specific standpoint.
K Williford's MoC4 presentation.

Goals: implement a subjective perspective for synthetically adaptive agents; in a way
compatible with axioms of consciousness (Alexander's axioms).
Rudrauf, Williford, Bennequin, Friston.
Mathematical model of embodied consciousness 2017; projective consciousness and
phenomenal selfhood 2018; moon illusion explained by the projective consciousness model
2020.

A step towards robotics.
World models, recalling the Bayesian brain hypothesis: adaptive systems have evolved to
preserve their integrity, which necessitates the prediction of environmental behaviour.
By forming hypotheses about the world and updating with new observations, to assist in
decision-making on how to act.
Friston active inference FKH06 DPS+20; Markov decision process and partially-observed.
Imbuing perspective-taking in social agents.
Plan in the future with respect to a prioris in environment.

Problem: associating actions and movement with perspective-taking (projective
transformation).
Subproblem: how to relate configuration and projective perspectives on the environment.
Impose a set of axioms in the agent's frame of reference.
Centre the subject after the transformation.
Preserve axes of Euclidean frame associated with the agent.
No points appear to be truly at infinity, only when subject directly represents a half
space.
Objects near to the agent have the same size in Euclidean frame.

Axioms impose a set of projective transformations.
Rudrauf et al 2022a, 2022b.

Model maladaptive behaviours.
Imagination and empathy: take others' perspective; simulate it.
Big claims but in a restricted setting.

Group-structured world models: incorporate actions and `perspectives' of agents within
the agent's internal space.
Reconstruct sensory input.
A way to put your own actions in the geometry of the space.
Plan the way you're moving inside of your space and how it will affect the world.
Think of it as a group acting on the space.

Rudrauf et al 2023.

What changes with respect to classical perspective?
POMDP don't have complete knowledge of environment via observations.
Specialization of POMDP: some actions relating to the agents, over which the agent has
control, affect its state space.

Euclidean vs projective case changes the behaviour of the agent to find an object.
Get closer to get more evidence about where object is.
Don't add any drive other than curiosity (mutual information).

POMDP is a particular form of causal model.
Group-structured world models are a form of inductive bias.
Equivariance to symmetries of objects that allow better prediction and action planning.

Challenges: difficult to have exact filtering and planning with theoretical guarantees.
Approximate method can be simplified.
Optimizing over possible group actions is also challenging, example learning by maximal
likelihood.
How to learn the possibility of taking different perspectives on the data?

Bayesian network: factorizing of probability distributions over variables (factor
graph, Markov random field).
Each variable is assumed to have a group acting on it, but the action is undetermined.
Fixed DAG but not forcing probability distributions on it.
Assumption about how group acts on model.
Composing action of group on factorized probability distribution.

Deep learning -- learn group-invariant transformations.
Not trying to be invariant \emph{to} something.
